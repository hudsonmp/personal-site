\documentclass{article}
\usepackage{amsmath}
\usepackage{amssymb}

\title{Orthogonal Curves}
\author{Hudson Mitchell-Pullman}
\date{2025}

\begin{document}

\maketitle

\section*{Problem}

Find a set of curves that is orthogonal to the set $\mathcal{F} = \{y = kx^2 : k \text{ is a constant}\}$.

\section*{Solution}

We let $\mathcal{F}$ and $\mathcal{G}$ denote a set of orthogonal curves, meaning that one or more of the curves in the sets intersect at a right angle.

We let $\mathcal{F}=\{y=kx^2 :k \text{ is a constant}\}$. Since $k$ is a constant that depends on the curves in $\mathcal{F}$, we want to express $k$ as a function of $y$ and $x$.

We rearrange $y=kx^2$ in terms of $k$, giving us $$k=\frac{y}{x^2}.$$ Next, we take the derivative of $y=kx^2$, giving us $$\frac{dy}{dx}=2kx$$ by implicit differentiation. 

Next, we make our substitution for $k$ (see above), giving us $$\frac{dy}{dx}=\frac{2y}{x}.$$ Next, we know the angle at the point of intersection of a curve's Normal is a right angle, meaning that to find the general equation for our orthogonal set, we should find a set $\mathcal{G}$ that is normal at a minimum of one point to our set $\mathcal{F}$. 

To do this, we take the negative reciprocal of our slope $\frac{dy}{dx}$. Since we have expressed $\frac{dy}{dx}$ in terms of $x$ and $y$ above, we write $$\frac{dy}{dx}=-\frac{x}{2y}.$$

We solve this first-order differential equation by separation of variables: 

\begin{align*}
2ydy&=-xdx \\
2\int ydy&=-\int xdx \\
y^2&=-\frac12x^2+C \\
C&=y^2+\frac12x^2 \\
C&=2y^2+x^2 \\
y&=\sqrt{\frac{C-x^2}{2}}
\end{align*} for a variable constant $C$.

Therefore, because the set $\mathcal{G} = \{y=\sqrt{\frac{C-x^2}{2}}\}$ is orthogonal to $\mathcal{F}$ at a minimum of one point (because it is a set of curves that are normal to curves in $\mathcal{F}$), we conclude that $$\mathcal{G} = \{y=\sqrt{\frac{C-x^2}{2}}\}.$$ \quad $\blacksquare$

\end{document}

