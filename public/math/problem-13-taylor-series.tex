\documentclass[12pt]{article}
\usepackage{amsmath}
\usepackage{amssymb}
\usepackage{geometry}
\usepackage{hyperref}
\geometry{margin=1in}

\title{Problem 13 – Taylor Series and Arcsin Function}
\author{Hudson Mitchell-Pullman}
\date{2025}

\begin{document}

\maketitle

\noindent\textit{Problem source: Art of Problem Solving (AoPS)}

\section*{Problem}

Let
\[f(x) = x + \frac{2}{3} x^3 + \frac{2 \cdot 4}{3 \cdot 5} x^5 + \dots + \frac{2 \cdot 4 \dotsm 2n}{3 \cdot 5 \dotsm (2n + 1)} x^{2n + 1} + \dotsb\]
on the interval $(-1,1)$ of convergence of the defining series.

\begin{enumerate}
    \item[(a)] Prove that $(1 - x^2) f'(x) = 1 + xf(x).$
    \item[(b)] Prove that $f(x) = \frac{\arcsin x}{\sqrt{1 - x^2}}.$
\end{enumerate}

\section*{Solution}

\subsection*{Part (a)}

We let $f(x)$ be a Taylor Series on the interval $(-1, 1)$ by denoting
\[f(x)=\sum_{n=0}^{\infty}c_nx^{2n+1},\]
such that $c_n=\frac{2\cdot4\cdot6\cdots2n}{3\cdot5\cdot7\cdots(2n+1)}$. We take the derivative of our sum with respect to $x$, giving us
\[f'(x)=\sum_{n=0}^{\infty}c_n(2n+1)x^{2n}.\]

We then multiply both sides by $(1-x^2)$ to get
\begin{align*}
(1-x^2)f'(x) &= (1-x^2)\sum_{n=0}^{\infty}c_n(2n+1)x^{2n}\\
&= \sum_{n=0}^{\infty}c_n(2n+1)x^{2n} - \sum_{n=0}^{\infty}c_n(2n+1)x^{2n+2}\\
&= \sum_{n=0}^{\infty}c_n(2n+1)x^{2n} - \sum_{n=1}^{\infty}c_{n-1}(2n-1)x^{2n}\\
&= c_0 + \sum_{n=1}^{\infty}[c_n(2n+1) - c_{n-1}(2n-1)]x^{2n}.
\end{align*}

Now we simplify the coefficient of $x^{2n}$ for $n \geq 1$:
\begin{align*}
c_n(2n+1) - c_{n-1}(2n-1) &= \frac{2\cdot4\cdots2n}{3\cdot5\cdots(2n+1)}(2n+1) - \frac{2\cdot4\cdots(2n-2)}{3\cdot5\cdots(2n-1)}(2n-1)\\
&= \frac{2\cdot4\cdots2n}{3\cdot5\cdots(2n-1)} - \frac{2\cdot4\cdots(2n-2)}{3\cdot5\cdots(2n-1)}\\
&= \frac{2\cdot4\cdots(2n-2)}{3\cdot5\cdots(2n-1)}\left(\frac{2n}{1} - 1\right)\\
&= \frac{2\cdot4\cdots(2n-2)}{3\cdot5\cdots(2n-1)}(2n-1)\\
&= c_{n-1}(2n-1).
\end{align*}

Wait, let me recalculate this more carefully. We have:
\begin{align*}
c_n(2n+1) - c_{n-1}(2n-1) &= \frac{2\cdot4\cdots2n}{3\cdot5\cdots(2n+1)}(2n+1) - \frac{2\cdot4\cdots(2n-2)}{3\cdot5\cdots(2n-1)}(2n-1)\\
&= \frac{2\cdot4\cdots2n}{3\cdot5\cdots(2n-1)} - \frac{(2\cdot4\cdots(2n-2))(2n-1)}{3\cdot5\cdots(2n-1)}\\
&= \frac{2\cdot4\cdots2n - (2\cdot4\cdots(2n-2))(2n-1)}{3\cdot5\cdots(2n-1)}\\
&= \frac{(2\cdot4\cdots(2n-2))[2n - (2n-1)]}{3\cdot5\cdots(2n-1)}\\
&= \frac{2\cdot4\cdots(2n-2)}{3\cdot5\cdots(2n-1)}\\
&= c_{n-1}.
\end{align*}

Therefore,
\begin{align*}
(1-x^2)f'(x) &= c_0 + \sum_{n=1}^{\infty}c_{n-1}x^{2n}\\
&= 1 + \sum_{n=1}^{\infty}c_{n-1}x^{2n}\\
&= 1 + \sum_{k=0}^{\infty}c_kx^{2k+2}\\
&= 1 + x\sum_{k=0}^{\infty}c_kx^{2k+1}\\
&= 1 + xf(x).
\end{align*}

\subsection*{Part (b)}

Let $g(x) = \sqrt{1 - x^2} f(x).$ Then by the product rule:
\begin{align*}
g'(x) &= \frac{d}{dx}[\sqrt{1 - x^2}] \cdot f(x) + \sqrt{1 - x^2} \cdot f'(x)\\
&= \frac{-x}{\sqrt{1 - x^2}} f(x) + \sqrt{1 - x^2} f'(x)\\
&= \frac{-xf(x) + (1-x^2)f'(x)}{\sqrt{1 - x^2}}\\
&= \frac{-xf(x) + 1 + xf(x)}{\sqrt{1 - x^2}}\\
&= \frac{1}{\sqrt{1 - x^2}}.
\end{align*}

Since $\frac{d}{dx}[\arcsin x] = \frac{1}{\sqrt{1-x^2}}$, we have that $g(x) = \arcsin x + C$ for some constant $C.$ 

To find $C$, we evaluate at $x = 0$:
\[g(0) = \sqrt{1 - 0^2} f(0) = 1 \cdot f(0) = f(0) = 0.\]

Also, $\arcsin(0) = 0$, so $g(0) = 0 + C$, which gives us $C = 0.$

Therefore, $g(x) = \arcsin x,$ which means
\[\sqrt{1 - x^2} f(x) = \arcsin x,\]
so
\[f(x) = \frac{\arcsin x}{\sqrt{1 - x^2}}.\]

\end{document}

