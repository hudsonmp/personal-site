\documentclass{article}
\usepackage{amsmath}
\usepackage{amssymb}

\title{Integral of Arctan}
\author{Hudson Mitchell-Pullman}
\date{2025}

\begin{document}

\maketitle

\noindent\textit{Problem source: Art of Problem Solving (AoPS)}

\section*{Problem}

Compute $\int \tan^{-1}x \,dx$.

\section*{Solution}

We must use integration by parts to solve this integral, but we notice that there are no obvious choices for $dv$ and $u$ since our integrand appears to consist of only one term. We set $u=\tan^{-1}x$ because we know $\frac{d}{dx}\tan^{-1}x=\frac{1}{x^2+1}$, forcing us to set $dv=dx$ since it is the only remaining term. We set $du=\frac{1}{x^2+1}dx$ and $v=x$ by the integral power rule.

Next, by integration by parts, we write
$$\int u\,dv=uv-\int v\,du \quad \rightarrow \quad \int\tan^{-1}x\,dx = x\tan^{-1}x-\int\frac{x}{x^2+1}dx.$$

We recognize that making a substitution will make our integral significantly easier to solve; we set $x=\tan\theta$ by substitution.

After our clever substitution, we have
$$\int\tan^{-1}x\, dx = \theta\tan\theta-\int\frac{\tan\theta}{\tan^2\theta+1} \sec^2\theta \,d\theta.$$

We may initially seem stuck, but we notice that when we divide our Fundamental Pythagorean Identity by $\cos^2\theta$, we get
$$\sin^2\theta+\cos^2\theta=1 \quad \rightarrow \quad \tan^2\theta + 1 = \sec^2\theta.$$

When we make this substitution using our trigonometric identity, we notice that our integrand simplifies nicely; our integral term can be written as
$$\int\frac{\tan\theta\sec^2\theta}{\sec^2\theta}d\theta=\int\tan\theta\,d\theta.$$

We will omit the proof for the antiderivative of tangent because we have proven it previously, but it's important to note that $\int\tan\theta\,d\theta=\log|\sec\theta|+C$ by substitution.

Combining this with our original equation, we have
$$\int\tan^{-1}x\, dx = \theta\tan\theta-\log|\sec\theta| +C.$$

We may think we're done, but we must remember that our original integral was written in terms of $x$, so our answer must also be in terms of $x$, not $\theta$. Fortunately, we can resolve this issue quickly.

We previously made the substitution $x=\tan\theta$. If we take the $\arctan$ of each side of the expression, we get $\theta=\tan^{-1}x$. Our final step is to rewrite our answer in terms of $x$, which we can do since we now know how to express $\theta$ in terms of $x$. We have
$$\int\tan^{-1}x\, dx = x\tan^{-1}x-\log|\sec(\tan^{-1}x)| + C,$$
which is our final answer in terms of $x$. \quad $\blacksquare$

\end{document}

